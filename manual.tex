\documentclass[a4paper]{article}

\usepackage[utf8]{inputenc}
\usepackage[main=czech,english]{babel}
\usepackage{graphicx}
\usepackage{xspace}
\usepackage{listigns}
\DeclareUnicodeCharacter{00A0}{ }

\title{Detekce aplikaci tvorici sitovy provoz}
\date{2016-11-14}
\author{Vratislav Hais, xhaisv00@stud.fit.vutbr.cz}

\begin{document}

	\pagenumbering{gobble}

	\begin{center}
		\begin{figure}[t!]
			\centering
			\includegraphics[width=\linewidth]{VUT_logo.png}
			\vspace{1in}
		\end{figure}

		\begin{center}
			\large
			\textsc{Vysoke Uceni Technicke v Brne}
			\textsc{Fakulta Informacnich Technologii}
			\vspace{2in}
		\end{center}
		{
			\huge
			Detekce aplikaci tvorici sitovy provoz
		}
	\end{center}

	\vspace*{\fill}
	{
		\large
		\foreignlanguage{english}{Autor: Vratislav Hais, xhaisv00}
	}


	\newpage
	\pagenumbering{arabic}
	\tableofcontents
	\newpage

	\section{Uvod}

	Tento dokument byl vytvoren jako dokumentace k projektu do predmetu ISA (Sitove aplikace a sprava siti).
	Zakladem programu je pouziti prikazu lsof (\emph{lsof -Pnl +M -i}). Vysledek lsofu je pomoci shellovskych prikazu a regularnich vyrazu zpracovan do pozadovaneho tvaru, ktery se bude posilat na syslog server. Komunikace se syslog serverem je zajistena pomoci BSD socketu. Poslane zpravy jsou kontrolovany utilitou netcat (\emph{nc -l -p514}).

	\section{BSD}

	BSD sockety jsou API pro komunikujici procesy. Jedna se o abstraktni datovou strukturu obsahujici potrebne udaje pro komunikaci. Pro vytvoreni socketu je potreba znat IP adresu a port, na kterem bude provadena komunikace.

	\subsection{Funkce pro praci s BSD sockety}

	\begin{itemize}
	\item socket() - vytvoreni noveho socketu. Je mozne vytvorit ruzne druhy socketu, coz se upresnuje v parametru funkce
	\item bind() - prideli adresu specifikovanou v \emph{addr} k socketu. Tradicne se teto operaci prezdiva "pojmenovani socketu"
	\item listen() - pouzivano na strane serveru. Prepne TCP socket do naslouchaciho modu
	\item accept() - pouzivano na strane serveru. Prijme obdrzeny pokus o pripojeni
	\item connect() - pouzivano na strane klienta. Prideli socketu volny port. V pripade TCP socketu se pokusi o navazani spojeni
	\item send(), sendto(), recv(), recvfrom(), write(), read() - pouzivaji se pro posilani/prijimani dat
	\item close() - v pripade TCP socketu ukonci spojeni. V jinem pripade uvolni systemem pridelenou pamet
	\item gethostbyname(), gethostbyaddr() - na zaklade jmena/ip adresy vraci informace potrebne k naplneni struktury \emph{hostent} potrebne k navazani spojeni
	\item select() - cekani na 1 nebo vic poskytnutych socketu az budou pripraveny ke cteni, zapisu nebo zda maji error
	\item getsockopt() - vraci hodnotu pozadovaneho prepinace zadaneho socketu
	\item setsockopt() - nastavi pozadovanou hodnotu prepinace zadaneho socketu
	\end{itemize}

	\section{Syslog}

	Syslog je standard pro záznam programových zpráv. Umožňuje oddělit:
	\begin{itemize}
	\item software generující zprávy
	\item od systému, který je ukládá
	\item softwaru, jenž poskytuje reporty a analýzy.
	\end{itemize}

	\section{lsof}

	\emph{lsof} je prikaz vypisujici seznam vsech otevrenych souboru, procesu a pripojeni.

	\section{netcat}

	Prikaz \emph{netcat} (nebo take \emph{nc}) je počítačoý program sloužící jako nástroj pro zápis a čtení ze TCP nebo UDP spojení.

	\section{Implementace}

	Aplikace je naprogramovana v jazyce C++ a je spustitelna na operacnim systemu Linux.

	\subsection{Argumenty}

	Uzivatel muze spustit aplikaci s libovolnym poradim argumentu. Toho je docileno cyklem, ktery zjistuje, o ktery argument se momentalne jedna a ulozi nasledujici hodnotu do struktury s argumenty. Pred ulozenim do struktury se kontroluje validita hodnoty, tzn ze zda se opravdu jedna o cislo (v pripade prepinace -i) nebo IP adresu (v pripade prepinace -s).
	\begin{itemize}
		\item -s parametr nasledovan IP adresou urcuje, na jake adrese bezi syslog server
		\item -f parametr nasledovan retezcem obsahujicim nazvy aplikaci, ktere ma uzivatel zajem sledovat
		\item -i parametr nasledovan cislem urcuje, v jakem intervalu se bude aktualizovat vypis
		\item -h, --help je samostatny parametr, ktery vypise napovedu k programu
	\end{itemize}
	Parametry -s, -f, -i jsou povinne. Pokud uzivatel nezada nektery z nich, tak se program ukonci s chybou.

	\subsection{funkce isNum() a isHexa()}

	Pomocne funkce slouzici pro kontrolu, zda je zkoumany retezec slozen z cisel desitkove soustavy (funkce isNum()) nebo z hexadecimalnich znaku (funkce isHexa())

	\subsection{funkce checkIp()}

	Jedna se o pomocnou funkci pri kontrole argumentu. Kdyz se kontroluje prepinac -s, zavola se tato funkce a preda se ji hodnota nasledovana za prepinacem. Funkce rozhodne, zda se jedna o IPv4 nebo IPv6, nastavi priznak ve strukture a podle vysledku zavola funkci checkIPv4() nebo checkIPv6(). Tyto funkce zkontroluji validitu zadane IP adresy. Pokud je adresa validni, tak se ulozi do struktury a pokracuje se zpracovavanim ostatnich argumentu.

	\subsection{funkce connection()}

	Funkce obstaravajici pripojeni k syslog serveru. Funkce vytvori socket, ktery se bude pouzivat k posilani dat na syslog server.

	\subsection{funkce getLine()}

	Pomocna funkce, ktera vraci prvni radek ze zadaneho retezce. Funkce je destruktivni, tudiz je vraceny radek zaroven z puvodniho retezce vymazan.

	\subsection{funkce isLocalhost()}

	Funkce zkouma, zda je v zadanem retezci vyskyt localhostu. Pokud ano, vraci hodnotu true.

	\subsection{funkce main()}

	Zde se odehrava veskera cinnost. Na zacatku funkce se vytvori retezec, ktery za pomoci funkce popen vrati data v pozadovanem tvaru.
	Zminovany retezec:
	\newline
	\begin{sloppypar}
		\begin{lstlistings}[breaklines]
			string command = "lsof -Pnl +M -i | grep ESTABLISHED | grep ";
			command += args.filter;  // pridani filtru na pozadovane aplikace 
			command += " | tr -s \" \" | cut -d\" \" -f 8,9,1 | awk '{ print $2 \" \" $3 \" \" $1}'";  // vyber pozadovanych sloupcu (nazev appky, IP, protokol) + prehazeni v pozadovanem poradi
			command += " | sed -r \"s/\\]|\\[|\\->/ /g\" | ";  // odmazani prebytecnych znaku + rozdeleni portu od IP v IPv6
			command += "sed -r \"s/(\\.[[:digit:]]+):/\\1 /g\" | "; // rozdeleni portu od IP v IPv4
			command += "sed -r \"s/ :/ /g\" | tr -s \" \" | uniq -u";
		\end{lstlistings}
	\end{sloppypar}
	Retezec je rozdeleny z duvodu prehlednosti a moznosti okomentovani jednotlivych casti. Po vytvoreni tohoto retezce zacina nekonecny cyklus. Na zacatku cyklu se vola drive zminovana funkce popen, ktera za pomoci vytvoreneho retezce v promenne \emph{command} vrati data v pozadovanem tvaru. Data jsou ulozena v promenne \emph{str} jako jeden celek. Tato data musime rozbit na jednotlive radky a ulozit do vectoru. K tomu slouzi jiz zminovana funkce getLine(), ktera se vola do doby, nez je promenna \emph{str} prazdna. Jednotlive vysledky teto funkce se ukladaji do vectoru \emph{toProcess}, ktery slouzi pouze jako pomocna promenna. Vysledek, ktery se bude posilat na syslog server a stdout bude ulozen ve vectoru \emph{result}. Pokud je \emph{result} prazdny (prvni pruchod nebo v predchozim pruchodu nebylo zadne spojeni), pak se obsah \emph{toProcess} rovnou ulozi do \emph{result}. Pokud prazdny neni, tak se spusti kontrola jednotlivych prvku. Pokud je nejaky zaznam v \emph{result} ktery neni v \emph{toProcess}, tak je vymazan. Pokud je stejny zaznam v obou vectorech, pak je vymazan z \emph{toProcess}. Pokud kontrolovany zaznam neni v \emph{result}, pak se zachova v \emph{toProcess}. Po ukonceni teto kontroly se zbyly obsah \emph{toProcess} ulozi do \emph{result} a vypisi se vsechny nove pridane prvky. 

	\subsection{Knihovny}

	V programu jsou pouzity nasledujici knihovny:
	\begin{sloppypar}
	stdio.h, stdlib.h, stdbool.h, cstring, iostream, sstream, ctype.h, vector, sys/socket.h, netinet/in.h, arpa/inet.h, netdb.h, unistd.h
	\end{sloppypar}

	\subsection{Preklad programu}

	Preklad je provaden prikazem:
	\begin{sloppypar}
		\begin{lstlistings}[breaklines]
			g++ appdetector.cpp -o appdetector
		\end{lstlistings}
	\end{sloppypar}

	\section{Pouziti aplikace}

	Spusteni:

	\textbf{./appdetector [-s IPaddress] [-f APP] [-i interval]}

	Priklad:

	./appdetector -s 127.0.0.1 -f "chrome" -i 10

	Mozny vystup (stdout, na syslog stejne akorat bez znaku konce radku):

	\begin{sloppypar}
		\begin{lstlistings}[breaklines]
			TCP 10.0.0.36 50862 31.13.84.36 443 chrome
			TCP 10.0.0.36 38829 31.13.93.7 443 chrome
			TCP 10.0.0.36 59914 64.233.184.189 443 chrome
			TCP 10.0.0.36 60583 31.13.84.8 443 chrome
			TCP 10.0.0.36 44137 64.233.166.188 5228 chrome
		\end{lstlistings}
	\end{sloppypar}

	\section{Metriky kodu}

	Pocet souboru: 2 (appdetector.cpp, Makefile)
	Pocet radku kodu: 388
	Velikost binarky: 40 960 bytů

	\begin{thebibliography}{9}
  	\bibitem{bsdsocketprogramming} 
  	M. Tim Jones.
  	\textit{BSD Sockets Programming From a Multi-Language Perspective}. 
  	Charles River, Massachusetts, 2003.