\documentclass[a4paper]{article}

\usepackage[utf8]{inputenc}
\usepackage[main=czech,english]{babel}
\usepackage{graphicx}
\usepackage{xspace}
\usepackage{listigns}
\DeclareUnicodeCharacter{00A0}{ }

\title{Detekce aplikací tvořící síťový provoz}
\date{2016-11-14}
\author{Vratislav Hais, xhaisv00@stud.fit.vutbr.cz}

\begin{document}

	\pagenumbering{gobble}

	\begin{center}
		\begin{figure}[t!]
			\centering
			\includegraphics[width=\linewidth]{VUT_logo.png}
			\vspace{1in}
		\end{figure}

		\begin{center}
			\large
			\textsc{Vysoke Učení Technické v Brně}
			\textsc{Fakulta Informačních Technologií}
			\vspace{2in}
		\end{center}
		{
			\huge
			Detekce aplikací tvořící síťový provoz
		}
	\end{center}

	\vspace*{\fill}
	{
		\large
		\foreignlanguage{english}{Autor: Vratislav Hais, xhaisv00}
	}


	\newpage
	\pagenumbering{arabic}
	\tableofcontents
	\newpage

	\section{Uvod}

	Tento dokument byl vytvořen jako dokumentace k projektu do předmětu ISA (Síťové aplikace a správa sítí).
	Základem programu je použití příkazu lsof (\emph{lsof -Pnl +M -i}). Výsledek lsofu je pomocí shellovských příkazů a regularních výrazů zpracován do požadovaného tvaru, který se bude posílat na syslog server. Komunikace se syslog serverem je zajištěna pomocí BSD socketu. Poslané zprávy jsou kontrolovány utilitou netcat (\emph{nc -l -p514}).

	\section{BSD}

	BSD sockety jsou API pro komunikujíci procesy. Jedná se o abstraktní datovou strukturu obsahujíci potřebné údaje pro komunikaci. Pro vytvoření socketu je potřeba znát IP adresu a port, na kterém bude prováděna komunikace.

	\subsection{Funkce pro práci s BSD sockety}

	\begin{itemize}
	\item socket() - vytvoření nového socketu. Je možné vytvořit různé druhy socketu, což se upřesňuje v parametru funkce
	\item bind() - přidělí adresu specifikovanou v \emph{addr} k socketu. Tradičně se této operaci přezdívá "pojmenování socketu"
	\item listen() - používáno na straně serveru. Přepne TCP socket do naslouchacího módu
	\item accept() - používáno na straně serveru. Přijme obdržený pokus o připojení
	\item connect() - používáno na straně klienta. Přidělí socketu volný port. V případě TCP socketu se pokusí o navázání spojení
	\item send(), sendto(), recv(), recvfrom(), write(), read() - používají se pro posílání/přijímání dat
	\item close() - v případě TCP socketu ukončí spojení. V jiném případě uvolní systémem přidělenou paměť
	\item gethostbyname(), gethostbyaddr() - na základě jména/ip adresy vrací informace potřebné k naplnění struktury \emph{hostent} potřebné k navázání spojení
	\item select() - čekání na 1 nebo víc poskytnutých socketů až budou připraveny ke čtení, zápisu nebo zda mají error
	\item getsockopt() - vrací hodnotu požadovaného přepínače zadaného socketu
	\item setsockopt() - nastaví požadovanou hodnotu přepínače zadaného socketu
	\end{itemize}

	\section{Syslog}

	Syslog je standard pro záznam programových zpráv. Umožňuje oddělit:
	\begin{itemize}
	\item software generující zprávy
	\item od systému, který je ukládá
	\item softwaru, jenž poskytuje reporty a analýzy.
	\end{itemize}

	\section{lsof}

	\emph{lsof} je příkaz vypisující seznam všech otevřených souborů, procesů a připojení.

	\section{netcat}

	Příkaz \emph{netcat} (nebo také \emph{nc}) je počítačoý program sloužící jako nástroj pro zápis a čtení ze TCP nebo UDP spojení.

	\section{Implementace}

	Aplikace je naprogramovaná v jazyce C++ a je spustitelná na operačním systému Linux.

	\subsection{Argumenty}

	Uživatel může spustit aplikaci s libovolným pořadím argumentu. Toho je docíleno cyklem, ktery zjišťuje, o který argument se momentálně jedná a uloží následující hodnotu do struktury s argumenty. Před uložením do struktury se kontroluje validita hodnoty, tzn že zda se opravdu jedná o číslo (v případě přepínače -i) nebo IP adresu (v případě přepínače -s).
	\begin{itemize}
		\item -s parametr následován IP adresou určuje, na jaké adrese běží syslog server
		\item -f parametr následován řetězcem obsahujícím názvy aplikací, které má uživatel zájem sledovat
		\item -i parametr následován číslem určuje, v jakém intervalu se bude aktualizovat výpis
		\item -h, --help je samostatný parametr, který vypíše nápovědu k programu
	\end{itemize}
	Parametry -s, -f, -i jsou povinné. Pokud uživatel nezadá některý z nich, tak se program ukončí s chybou.

	\subsection{funkce isNum() a isHexa()}

	Pomocné funkce sloužící pro kontrolu, zda je zkoumaný řetězec složen z čísel desítkové soustavy (funkce isNum()) nebo z hexadecimálních znaků (funkce isHexa())

	\subsection{funkce checkIp()}

	Jedná se o pomocnou funkci při kontrole argumentu. Když se kontroluje přepínač -s, zavolá se tato funkce a předá se ji hodnota následovaná za přepínačem. Funkce rozhodne, zda se jedna o IPv4 nebo IPv6, nastaví příznak ve struktuře a podle výsledku zavolá funkci checkIPv4() nebo checkIPv6(). Tyto funkce zkontrolují validitu zadane IP adresy. Pokud je adresa validní, tak se uloží do struktury a pokračuje se zpracováváním ostatních argumentů.

	\subsection{funkce connection()}

	Funkce obstarávající připojení k syslog serveru. Funkce vytvoří socket, který se bude používat k posílání dat na syslog server.

	\subsection{funkce getLine()}

	Pomocná funkce, která vrací první řádek ze zadaného řetězce. Funkce je destruktivní, tudíž je vrácený řádek zároveň z původního řetězce vymazán.

	\subsection{funkce isLocalhost()}

	Funkce zkoumá, zda je v zadaném řetězci výskyt localhostu. Pokud ano, vrací hodnotu true.

	\subsection{funkce main()}

	Zde se odehrává veškerá činnost. Na začátku funkce se vytvoří řetězec, který za pomocí funkce popen vrátí data v požadovaném tvaru.
	Zmiňovaný řetězec:
	\newline
	\begin{sloppypar}
		\begin{lstlistings}[breaklines]
			string command = "lsof -Pnl +M -i | grep ESTABLISHED | grep ";
			command += args.filter;  // pridani filtru na pozadovane aplikace 
			command += " | tr -s \" \" | cut -d\" \" -f 8,9,1 | awk '{ print $2 \" \" $3 \" \" $1}'";  // vyber pozadovanych sloupcu (nazev appky, IP, protokol) + prehazeni v pozadovanem poradi
			command += " | sed -r \"s/\\]|\\[|\\->/ /g\" | ";  // odmazani prebytecnych znaku + rozdeleni portu od IP v IPv6
			command += "sed -r \"s/(\\.[[:digit:]]+):/\\1 /g\" | "; // rozdeleni portu od IP v IPv4
			command += "sed -r \"s/ :/ /g\" | tr -s \" \" | uniq -u";
		\end{lstlistings}
	\end{sloppypar}
	Řetězec je rozdělený z důvodu přehlednosti a možnosti okomentování jednotlivých častí. Po vytvoření tohoto řetězce začíná nekonečný cyklus. Na začátku cyklu se volá dříve zmiňovaná funkce popen, která za pomoci vytvořeného řetězce v proměnné \emph{command} vrátí data v požadovaném tvaru. Data jsou uložena v proměnné \emph{str} jako jeden celek. Tato data musíme rozbít na jednotlivé řádky a uložit do vectoru. K tomu slouží již zmiňovaná funkce getLine(), která se volá do doby, než je proměnná \emph{str} prázdná. Jednotlivé výsledky této funkce se ukládají do vectoru \emph{toProcess}, který slouží pouze jako pomocná proměnná. Výsledek, který se bude posílat na syslog server a stdout bude uložen ve vectoru \emph{result}. Pokud je \emph{result} prázdný (první průchod nebo v předchozím průchodu nebylo žádné spojení), pak se obsah \emph{toProcess} rovnou uloží do \emph{result}. Pokud prázdný není, tak se spustí kontrola jednotlivých prvků. Pokud je nějaký záznam v \emph{result} který není v \emph{toProcess}, tak je vymazán. Pokud je stejný záznam v obou vectorech, pak je vymazán z \emph{toProcess}. Pokud kontrolovaný záznam není v \emph{result}, pak se zachová v \emph{toProcess}. Po ukončení této kontroly se zbylý obsah \emph{toProcess} uloží do \emph{result} a vypíší se všechny nově přidané prvky. 

	\subsection{Knihovny}

	V programu jsou použity následující knihovny:
	\begin{sloppypar}
	stdio.h, stdlib.h, stdbool.h, cstring, iostream, sstream, ctype.h, vector, sys/socket.h, netinet/in.h, arpa/inet.h, netdb.h, unistd.h
	\end{sloppypar}

	\subsection{Překlad programu}

	Překlad je prováděn příkazem:
	\begin{sloppypar}
		\begin{lstlistings}[breaklines]
			g++ appdetector.cpp -o appdetector
		\end{lstlistings}
	\end{sloppypar}

	\section{Použití aplikace}

	Spuštění:

	\textbf{./appdetector [-s IPaddress] [-f APP] [-i interval]}

	Příklad:

	./appdetector -s 127.0.0.1 -f "chrome" -i 10

	Možný výstup (stdout, na syslog stejně, akorát bez znaku konce řádku):

	\begin{sloppypar}
		\begin{lstlistings}[breaklines]
			TCP 10.0.0.36 50862 31.13.84.36 443 chrome
			TCP 10.0.0.36 38829 31.13.93.7 443 chrome
			TCP 10.0.0.36 59914 64.233.184.189 443 chrome
			TCP 10.0.0.36 60583 31.13.84.8 443 chrome
			TCP 10.0.0.36 44137 64.233.166.188 5228 chrome
		\end{lstlistings}
	\end{sloppypar}

	\section{Metriky kódu}

	Počet souborů: 2 (appdetector.cpp, Makefile)
	Počet řádků kódu: 388
	Velikost binárky: 40 960 bytů

	\begin{thebibliography}{9}
  	\bibitem{bsdsocketprogramming} 
  	M. Tim Jones.
  	\textit{BSD Sockets Programming From a Multi-Language Perspective}. 
  	Charles River, Massachusetts, 2003.